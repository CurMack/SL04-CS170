\documentclass[11pt]{article}

\usepackage{amsmath,textcomp,amssymb,geometry,graphicx,enumerate,tikz}

\makeatletter
\newif\if@restonecol
\makeatother
\let\algorithm\relax
\let\endalgorithm\relax
\usepackage[linesnumbered,ruled,vlined]{algorithm2e}

\def\Class{CS170}
\def\Name{CurMack}  % Your name
\def\Homework{6} % Number of Homework
\def\Session{Spring 2022}

\title{\Class-- \Session --- Homework \Homework}
\author{\Name}
\markboth{\Class--\Session\  Homework \Homework\ \Name}{\Class--\Session\ Homework \Homework\ \Name}
\pagestyle{myheadings}
\date{\today}

\def\mi{\textbf{Main Idea}: }
\def\co{\textbf{Correctness}: }
\def\rt{\textbf{Runtime}: }
\def\ft{\textbf{Function}: }
\def\bc{\textbf{Base Case}: }
\def\sp{\textbf{Space}: }
\def\rc{\textbf{Recurrence}: }


\newenvironment{qparts}{\begin{enumerate}[(a)]}{\end{enumerate}}
\def\endproofmark{$\Box$}
\newenvironment{proof}{\par{\bf Proof}:}{\endproofmark\smallskip}

\textheight=9in
\textwidth=6.5in
\topmargin=-.75in
\oddsidemargin=0.25in
\evensidemargin=0.25in

\begin{document}
	\maketitle
	\section*{Chain Matrix Multiplication}
	Multiplying an $m \times n$ matrix by an $n \times p$ matrix takes $mnp$ multiplications. How do we determine the optimal order, if we want to compute $A_1 \times A_2 \times \dots \times A_n$, where the $A_i$’s are matrices with dimensions $m_0 \times m_1, m_1 \times m_2,\dots, m_{n-1} \times m_n$, respectively? \\
	\ft For $1\leq i \leq j\leq n$, define:
	\begin{center}
		$C(i, j) = $ minimum cost of multiplying $A_i \times A_{i+1} \times \dots \times A_j$
	\end{center}
	\bc when $i = j$, $C(i, i) = 0$\\
	\rc 
	$$C(i, j) = \min_{i \leq k < j}\{C(i, k) + C(k+1, j) + m_{i-1}\cdot m_k\cdot m_j\}$$
	\mi \\
	\begin{algorithm}[H]
		\For{$i = 1: n$}{
			$C(i, i) \leftarrow 0$
		}
		\For{$s = 1: n-1$}{
			\For{$i = 1: n-s$}{
				$j \leftarrow i + s$\\
				$C(i, j) \leftarrow \min\{C(i, k) + C(k+1, j) + m_{i-1}\cdot m_k\cdot m_j : i\leq k < j\}$
			}
		}
		\Return{$C(1, n)$}
	\end{algorithm}

	\rt $\mathcal{O}(n^3)$

	\section*{2  Egg Drop Revisited}
	\begin{qparts}
		\item 
		$$M(d, k) = M(d-1, k-1) + M(d-1, k) + 1$$
		The highest floor we can drop the first egg from is $M(d-1, k-1) +1$, if the egg breaks, we can still solve the problem with the remaining $d-1$ drops and $k-1$ eggs. If the egg doesn't break, now we have $d-1$ drops and $k$ eggs, we can at most solve $M(d-1, k)$ floors. So the  the maximum number of floors for which we can always find $l$ in at most $d$ drops using $k$ eggs is $M(d-1, k-1) + M(d-1, k) + 1$.  
		
		\item 
		For base cases, we take $M(0, k) = 0$ for any $k$ and $M(d, 0) = 0$ for any $d$. Starting with $d = 1$, we compute $M(d, x)$ for all, $1 \leq x \leq k$, and do so again for increasing values of $d$, up until we compute $M(d, x)$ for all $1 \leq x \leq k$. We return $M(d, k)$.\\
		\rt $\mathcal{O}(dk)$ (we compute $dk$ subproblems, each of which takes $\mathcal{O}(1)$ time)
		
		\item 
		Similarly, Starting with $d = 1$, we compute $M(d, x)$ for all, $1 \leq x \leq k$, and do so again for increasing values of $d$, up until we firstly compute $M(d, k)\geq n$ , then we return the value $d$ as $f(n, k)$.
		
		\item Notices that $d$ will always be at most $n$, since each floor will have at most $1$ drop for the optimal solution. Since the original runtime is $\mathcal{O}(dk)$ and $d \leq n$:\\
		\rt $\mathcal{O}(nk)$
		
		\item we only need to store $M(d - 1, x)$ and $M(d, x)$ for all $x$, i.e. we only ever need to store $\mathcal{O}(k)$ values. In particular, after computing $M(d, x)$ for all $x$, we can delete our stored values of $M(d - 1, x)$.
	\end{qparts}
	
	\section*{3 Knightmare} 
	\begin{qparts}
		\item Use $M$-bit string to represent a valid configuration of knights on a single row, there are $2^M$ representations. We will solve the $N\times M$ chessboard from the subproblem of size $N-1\times M$, since the $n$th row configuration depends on the $n-1$th row and $n-2$th row. \\
		\ft 
		\begin{center}
			$K(n, u, v) = $ the number of ways in an $n$-row board, \\
			$u$ be the specific configuration of the $n$th row, \\
			$v$ be the specific configuration of the $(n-1)$th row.
		\end{center} 
		
		\item \bc For all configurations $u$ and $v$ (no matter valid or not) :
		$$K(2, u, v) = \begin{cases}
			1 & \text{if valid}\\
			0 & \text{otherwise}
		\end{cases}$$
		
		\rc \\
		$$K(n, v, w) = \sum_{\text{all valid } u, v, w}K(n-1, u, v)$$
		return $\sum_{\text{all valid } u, v}K(n, u, v)$	
		
		\item \co for base case, we will brute force $n = 2$ rows, which s correct. If we have valid configuration $K(n-1, u, v)$, then for the $n$th row, we check last $3$ rows $u, v, w$ to see if they are valid and add all configuration to the $n^{th}$ row solution to solve $K(n, v, w)$, which is correct.
		
		\item \rt $\mathcal{O}(2^{3M}\cdot N\cdot M)$, we have $\mathcal{O}(N)$ rows, $\mathcal{O}(2^{3M})$ subproblems, each has $\mathcal{O}(M)$ to check.\\
		\sp $\mathcal{O}(N\cdot 2^{2M})$, we have $\mathcal{O}(N)$rows $\cdot$ $\mathcal{O}(2^{2M})$ subproblems per row.\\
		We only need to store the last two row, so the space we need is: $\mathcal{O}(2^{2M})$.
	\end{qparts}
	
	\section*{4 Balloon Popping Problem}
	\begin{qparts}
		\item Like matrix chain multiplication:\\
		\ft 
		\begin{center}
			$C(i, j) = $ maximum amount of noise produced by popping balloons  $i, i+1, \dots, j$
		\end{center}
		
		\item
		\bc $s_0 = 1$ and $s_{n+1} = 1$, assuming the input is from $1$ to $n$.\\
		For $i = 1$ to $n$, $C(i, i) = s_{i-1} \times s_{i} \times s_{i+1}$\\
		If $i > j$, $C(i, j) = 0$.
		
		\item
		\rc
		$$C(i, j) = \max_{i \leq k \leq j}\{C(i, k) + C(k+1, j) + s_{i-1}\cdot s_k\cdot s_{j+1}\}$$
		$k$ is the last balloon to pop in subset $i, i+1, \dots, j$.\\
		Finally return $C(1, n)$.\\
		\rt $\mathcal{O}(n^3)$
	\end{qparts}

	\section*{5  Paper Cutting}
	\begin{qparts}
		\item \ft 
		\begin{center}
			$B(i_1, j_1, i_2, j_2) = $ the minimum number of cuts \\
			needed to separate the matrix $A[i_1\dots i_2, j_1\dots j_2]$
		\end{center}
		
		\item \rc
		$$B(i_1, j_1, i_2, j_2) = \min \begin{cases}
			0  \ \ \  \text{if all entries in} A[i_1\dots i_2, j_1\dots j_2] \text{ are equal}\\
			1 + B(i_1, j_1, i_1+k, j_2) + B(i_1+k+1, j_1, i_2, j_2) & \text{for any }k \in \{1, \dots , i_2-i_1\}\\
			 1 + B(i_1, j_1, i_2, j_1+k) + B(i_1, j_1+k +1, i_2, j_2) & \text{for any }k \in \{1, \dots , j_2-j_1\}\\
		\end{cases}$$
		\bc set all single-square pieces to be $0$.
		
		\item \rt $\mathcal{O}((m+n)m^2n^2)$\\
		We have $\mathcal{O}(m^2n^2)$ total subproblems.  For each subproblem, we examine up to $m$ possible choices for horizontal splits, and $n$ possible choices for vertical splits, which takes $\mathcal{O}(n + m)$ time. We can precompute the purities of every single possible subrectangle and store it in a table. So to solve our recurrence relation, if we can determine purity/impurity in $\mathcal{O}(1)$ time, then we can reach an overall time of $\mathcal{O}((m+n)m^2n^2)$.
	\end{qparts}
\end{document}