% Search for all the places that say "PUT SOMETHING HERE".

\documentclass[11pt]{article}

\usepackage{amsmath,textcomp,amssymb,geometry,graphicx,enumerate}

\makeatletter
\newif\if@restonecol
\makeatother
\let\algorithm\relax
\let\endalgorithm\relax
\usepackage[linesnumbered,ruled,vlined]{algorithm2e}

\def\Name{CurMack}  % Your name
\def\Homework{1} % Number of Homework
\def\Session{Spring 2022}


\title{CS170--Spring 2022 --- Homework \Homework}
\author{\Name}
\markboth{CS170--\Session\  Homework \Homework\ \Name}{CS170--\Session\ Homework \Homework\ \Name}
\pagestyle{myheadings}
\date{}

\newenvironment{qparts}{\begin{enumerate}[(a)]}{\end{enumerate}}
\def\endproofmark{$\Box$}
\newenvironment{proof}{\par{\bf Proof}:}{\endproofmark\smallskip}

\textheight=9in
\textwidth=6.5in
\topmargin=-.75in
\oddsidemargin=0.25in
\evensidemargin=0.25in


\begin{document}
	\maketitle
	
	% Collaborators: PUT SOMETHING HERE (LIST OF YOUR COLLABORATORS, OR WRITE NONE)
	
	\section*{Master Theorem}
	If $T(n) = aT( \lceil n/b  \rceil) + \mathcal{O}(n^d)$ for some constants $a>0, b>1$, and $d\geq 0$, then:
	$$T(n) = \begin{cases}
		\mathcal{O}(n^d) & \text{if } d > \log_ba\\
		\mathcal{O}(n^d\log n) & \text{if } d = \log_ba\\
		\mathcal{O}(n^{\log_ba}) & \text{if } d < \log_ba
	\end{cases}$$
	
	\vspace{0.7cm}
	
	\section*{2. Recurrence Relations}
	\begin{qparts}
		\item $T(n) = 4T(n/2) + 42n$\\
		Use Mater Theorem: $a = 4, b = 2, d = 1$, since $d < \log_ba$, so:
		$$T(n) = \Theta(n^{\log_ba}) = \Theta(n^2)$$
		
		\item $T(n) = 4T(n/3) + n^2$\\
		Use Mater Theorem: $a = 4, b = 3, d = 2$, since $d > \log_ba$, so:
		$$T(n) = \Theta(n^{d}) = \Theta(n^2)$$
		
		\item $T(n) = T(3n/5) + T(4n/5)$(We have $T(1) = 1$)\\
		Notice that $5^2 = 3^2 + 4^2$, so suppose $T(n) = n^2$, then:
		$$T(n) = T(3n/5) + T(4n/5) = (\frac{3}{5}n)^2 + (\frac{4}{5}n)^2 = n^2$$
		Therefore the above recurrence is $T(n) = n^2$, which means $T(n) = \Theta(n^2)$
	\end{qparts}
	
	\vspace{0.7cm}
	
	\section*{3. Computing Factorials}
	\begin{qparts}
		\item (Note: You may not use Stirling’s formula)\\
		Since: $\log(N!) \leq \log (N^N) = N\log N $, so it's $\mathcal{O}(N\log N)$.\\
		Then: $\log(N!) \geq \log((\frac{N}{2})^{N/2}) = \frac{N}{2}\cdot (\log(N)-1)$, so it's also $\Omega(N\log N)$.\\
		Therefore, the number has $\Theta(N\log N)$ bits.
		
		\item 
		we compute  $N!$ naively.\\
		\begin{algorithm}[H] 
			\caption{factorial($N$)}
			% \KwIn{}
			% \KwOut{}
			$f \leftarrow 1$\\
			\For{$i = 2:N$}
			{
				$f \leftarrow f \cdot i$
			}
			\Return $f$ 
		\end{algorithm}
		And the runtime will be $\Theta(N^2\log^2 N)$.
	\end{qparts}
	
	\vspace{0.7cm}
	
	\section*{4. Decimal to Binary}
	Like Karatsuba's algorithm , we can take an n-digit number $x$ as $10^{n/2} \cdot a + b$ where $a, b, 10^{n/2}$ are 3 $n/2$-digit numbers. So the algorithm will recursively compute the binary representation of $a, b$ and $10^{n/2}$, then the algorithm will take one multiplication and one addition.Since the multiplication takes $\mathcal{O}(n^{\log_23}) $ by the Karatsuba's algorithm, therefore:
	$$T(n) = 3T(n/2) + \mathcal{O}(n^{\log_23})$$
	According to the Master Theorem, it has solution $T(n) = \mathcal{O}(n^{\log_23}\log n)$, which doesn't take much more time than Karatsuba's algorithm.
	
	\vspace{0.7cm}
	
	\section*{5. Pareto Optimality}
	\begin{qparts}
		\item 
		Firstly sort the point array by x-coordinate. Then split the array into 2 subset by x-coordinate. Let $L$ be the left half and $R$ be the right half. And $L', R'$ be the sets of Pareto-optimal points returned. Since every points in $R'$ is Pareto-optimal, and for each point in $L'$, if its $y$-coordinate is lager then the largest $y$-coordinate $y_{max}$ in $R'$, it is Pareto optimal. So remove the points with lower $y$-coordinate in $L'$ and return the union of $L'$ and $R'$. \\
		Since the scan takes linear time, so:
		$$T(n) = 2T(n/2) + \mathcal{O}(n)$$
		According to the Master theorem , the runtime is $\mathcal{O}(n\log n)$.
		\item Pseudocode:\\
		\begin{algorithm}[H]
			\caption{Pareto Optimality(P)}
			\KwIn{P is a piont array preprocessed by sorting in $x$-coordinate.}
			$P' \leftarrow \phi$\\
			split $P$ into $L$ and $R$ by $x$-coordinate.\\
			$L' \leftarrow \textbf{PO}(L), R' \leftarrow \textbf{PO}(R)$\\
			$P' \leftarrow P' \cup R'$\\
			$y_{max} \leftarrow$ the maximum $y$-value in $R'$\\
			\For{$(x_i, y_i) \in L'$}{
				\If{$y_i > y_{max}$}{
					$P' \leftarrow P' \cup (x_i, y_i)$
				}
			}
			\Return{P'}
		\end{algorithm}
	\end{qparts}
\end{document}