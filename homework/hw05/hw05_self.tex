\documentclass[11pt]{article}

\usepackage{amsmath,textcomp,amssymb,geometry,graphicx,enumerate}

\makeatletter
\newif\if@restonecol
\makeatother
\let\algorithm\relax
\let\endalgorithm\relax
\usepackage[linesnumbered,ruled,vlined]{algorithm2e}

\def\Class{CS170}
\def\Name{CurMack}  % Your name
\def\Homework{5} % Number of Homework
\def\Session{Spring 2022}

\title{\Class-- \Session --- Homework \Homework}
\author{\Name}
\markboth{\Class--\Session\  Homework \Homework\ \Name}{\Class--\Session\ Homework \Homework\ \Name}
\pagestyle{myheadings}
\date{\today}

\def\mi{\textbf{Main Idea}: }
\def\co{\textbf{Correctness}: }
\def\rt{\textbf{Runtime}: }

\newenvironment{qparts}{\begin{enumerate}[(a)]}{\end{enumerate}}
\def\endproofmark{$\Box$}
\newenvironment{proof}{\par{\bf Proof}:}{\endproofmark\smallskip}

\textheight=9in
\textwidth=6.5in
\topmargin=-.75in
\oddsidemargin=0.25in
\evensidemargin=0.25in

\begin{document}
	\maketitle
	
	\section*{Knapsack with repetition}
	Define:
	\begin{center}
		$K(w) = $ maximum value achievable with a knapsack of capacity $w$.
	\end{center}
	express this in terms of smaller subproblems:
	$$k(w) = \max_{i: w_i \leq w}\{K(w - w_i) + v_i\}$$
	where as usual our convention is that the maximum over an empty set is $0$.\\
	So the algorithm:\\
	\begin{algorithm}[H]
		$K(0) \leftarrow 0$\\
		\For{$w = 1: W$}{
			$K(w) \leftarrow \max\{K(w - w_i) + v_i : w_i \leq w\}$
		}
		\Return{$K(W)$}
	\end{algorithm}
	\rt  $\mathcal{O}(nW)$
	
	\section*{Knapsack without repetition}
	Definr:
	\begin{center}
		$K(w, j) = $ maximum value achievable using a knapsack of capacity $w$ and items $1, \dots j$.
	\end{center}
	The answer we seek is $K(W, n)$.\\
	express a subproblem $K(w, j)$ in terms of smaller subproblems:
	$$K(w, j) = \max\{K(w - w_j, j-1) + v_j, K(w, j-1)\}.$$
	So the algorithm:\\
	\begin{algorithm}[H]
		Initialize all $K(0, j) \leftarrow 0$ and all $K(w, 0) \leftarrow 0$\\
		\For{$j = 1: n$}{
			\For{$w = 1: W$}{
				\If{$w_j > w$}{
					$K(w, j) \leftarrow K(w, j-1)$
				}
				\Else{
					$K(w, j) \leftarrow \max\{K(w, j-1), K(w - w_j, j-1) + v_j\}$
				}
			}
		}
		\Return{$K(W, n)$}
	\end{algorithm}
	\rt $\mathcal{O}(nW)$
	
	\section*{2  Copper Pipes}
	\mi Knapsack with repetition:\\
	\begin{algorithm}[H]
		\caption{\texttt{cutPipe}$(price, n)$}
		$val(0) \leftarrow 0$\\
		\For{$w = 1: n$}{
			$max\_val \leftarrow 0$\\
			\For{$i = 1: w$}{
				$max\_val \leftarrow \max\{val(w - i) + price(i), max\_val\}$
			}
			$val(i) \leftarrow max\_val$
		}
		\Return{$val(n)$}
	\end{algorithm}
	\rt  $\mathcal{O}(n^2)$
	
	\section*{3 Egg Drop}
	\begin{qparts}
		\item 
		$f(1, k) = 1$: drop the egg from the single floor.\\
		$f(0, k) = 0$: there is only one possible value for $l$.\\
		$f(n, 1) = n$:  drop it from floor $1$ and going up, until it breaks.\\
		$f(n, 0) = \infty$:  the problem is unsolvable if we have no eggs to drop.
		
		\item the recurrence relation is:
		$$f(n, k) = \min_{x\in \{1\dots n\}} \max\{f(x-1, k-1), f(n-x, k)\}$$
		Suppose we drop at floor $x$:
		\begin{itemize}
			\item breaks: we know $l \in [0, x-1]$, and we have $k-1$ eggs $\Rightarrow f(x-1, k-1)$
			
			\item doesn't break: now $l \in [x, n]$, $k$ eggs $\Rightarrow f(n-x, k)$
		\end{itemize}
		And we pick the beat $x$
	\end{qparts}
\end{document}